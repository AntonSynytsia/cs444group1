\documentclass[onecolumn, draftclsnofoot, 10pt]{IEEEtran}

\usepackage{graphicx}
\usepackage{url}
\usepackage{setspace}

\usepackage{amssymb}
\usepackage{amsmath}
\usepackage{amsthm}
\usepackage{alltt}
\usepackage{color}
\usepackage{enumitem}
\usepackage{textcomp}
\usepackage{cite}
\usepackage{listings}

\usepackage{lmodern}
\usepackage[hidelinks]{hyperref}
\usepackage[normalem]{ulem}

\usepackage[utf8]{inputenc}
\usepackage[english]{babel}
\usepackage{lineno}
\usepackage{upquote}
\usepackage{fvextra}
\usepackage{fancyvrb}
\usepackage{minted}
\usepackage{longtable,hyperref}

\usepackage{geometry}
\geometry{textheight=9.5in, textwidth=7in}

%\parindent = 0.0 in
%\parskip = 0.1 in

%pull in the necessary preamble matter for pygments output
%\input{../ThirdParty/pygments.tex}

\newcommand{\longtableendfoot}{Please continue at the next page}

\begin{document}
\begin{titlepage}
\pagenumbering{gobble}
\begin{singlespace}
\centering
\scshape{
    \huge{Writeup 3}\par
    \vspace{.5in}
    \large{CS 444}\par
    \large{November 7, 2018}\par
    \vspace{.5in}
    \large{Anton Synytsia, Eytan Brodsky, David Jansen}\par
    \vspace{.5in}
    \vfill
}
%\begin{abstract}
%\end{abstract}
\end{singlespace}
\end{titlepage}
\newpage
\pagenumbering{arabic}
\tableofcontents
\clearpage

\section{Setup}

\subsection{Downloading Raspberry Pi Linux Kernel}
The following steps were performed to setup Raspberry Pi Linux kernel, on \texttt{OS2}:
\inputminted[breaklines]{bash}{cs1.sh}


\subsection{Preparing SD Card}
We chose Raspbian Stretch Lite image for our Raspberry Pi kernel. The following was done to setup Raspbian on SD card:
\begin{enumerate}
\item Download and extract \texttt{2018-10-09-raspbian-stretch-lite.zip} from \url{https://www.raspberrypi.org/downloads/raspbian/}.
\item Download and install Etcher from \url{https://www.balena.io/etcher/}.
\item Insert SD card into laptop.
\item Start Etcher and do the following:
    \begin{enumerate}
    \item Set image to \texttt{2018-10-09-raspbian-stretch-lite.img}
    \item Set drive to the SD Card
    \item Click ``Flash!``
    \end{enumerate}
\item Once the setup is complete, navigate to the SD Card drive and use text editor to append the following to \texttt{config.txt}:
\inputminted[breaklines]{bash}{cs2.sh}
\end{enumerate}


\subsection{Testing Raspbian}
The following steps, heavily based on adafruit guide, were taken to initiate a Raspbian serial console session:
\begin{description}
\item Install Prolific Chipset and driver \cite{adafruit1}.
\item Connect black, white, and green wires to the outer pins $3$, $4$, and $5$ respectively \cite{adafruit1}.
\item Leave red wire unpinned, as the a separate power adapter is used \cite{adafruit1}. It is important that only one power source is used as the board can get damaged \cite{adafruit1}.
\item Insert the Micro SD Card into Raspberry Pi.
\item Insert TTL Serial Cable USB into Laptop.
\item Insert the power adapter into Raspberry Pi.
\item Start Putty and do the following:
\begin{description}
\item Set \textit{Connection type} to serial mode.
\item Set \textit{Serial line} to \texttt{COM6}; {COM6} here refers to our, particular cable, which on Windows version we determined by accessing \textit{Device Manager} \cite{adafruit1}.
\item Set \textit{Speed} to $115200$ \cite{adafruit1}.
\item Click \textit{Open}.
\end{description}
\item Within the serial console, press the \textit{RETURN} key to activate communications \cite{adafruit1}.
\end{description}


\subsection{Verifying Build Environment}



\section{Morse Code LED Blinking}



\section{Concurrency}


\clearpage
\medskip
\bibliographystyle{IEEEtran}
\bibliography{ref}

\end{document}
