\documentclass[letterpaper,10pt]{article}

\usepackage{graphicx}
\usepackage{amssymb}
\usepackage{amsmath}
\usepackage{amsthm}
\usepackage{alltt}
\usepackage{color}
\usepackage{url}
\usepackage{geometry}

% set margins to 0.75in
\geometry{textheight=9.5in, textwidth=7in}
%letter is 11 x 8.5 w/out margins

%random comment

\newcommand{\cred}[1]{{\color{red}#1}}
\newcommand{\cblue}[1]{{\color{blue}#1}}

\newcommand{\toc}{\tableofcontents}

%\usepackage{hyperref}

\def\name{CS444 Group1}


%pull in the necessary preamble matter for pygments output
\input{pygments.tex}

%% The following metadata will show up in the PDF properties
% \hypersetup{
%   colorlinks = false,
%   urlcolor = black,
%   pdfauthor = {\name},
%   pdfkeywords = {cs311 ``operating systems'' files filesystem I/O},
%   pdftitle = {CS 311 Project 1: UNIX File I/O},
%   pdfsubject = {CS 311 Project 1},
%   pdfpagemode = UseNone
% }

\parindent = 0.0 in
\parskip = 0.1 in

\begin{document}

%\tableofcontents

\section{WriteUp 1}

\subsection{Command Log}
{\bfseries Logging into OS2}\\
1). ssh os2.engr.oregonstate.edu\\\\
{\bfseries Creating Group Folder}\\
1). cd /scratch/fall/2018\\
2). mkdir group1\\\\
{\bfseries Cloning Repository}\\
1). cd /scratch/fall2018/group1/\\
2). git clone git://git.yoctoproject.org/linux-yocto-3.19\\
3). git checkout tags/v3.19.2\\\\
{\bfseries Copying Files}\\
1). cp -R /scratch/files. /scratch/fall2018/group1\\\\
{\bfseries Setting Up Environment}\\
1). cd /scratch/fall2018/group1/\\
2). source environment-setup-i586-poky-linux\\
3). qemu-system-i386 -gdb tcp::5501 -S -nographic -kernel bzImage-qemux86.bin -drive\\ file=core-image-lsb-sdk-qemux86.ext4,if=virtio -enable-kvm -net none -usb -localtime\\
--no-reboot --append "root=/dev/vda rw console=ttyS0 debug"\\

This will hang, which it should; to unhang the process, follow the steps below.

{\bfseries Debugging}\\
Start another terminal in OS2 and run the following commands:\\
1). gdb\\
2). target remote tcp:5501\\
3). continue\\\\
{\bfseries Part 2: Testing Toolchain}\\
1). cd /scratch/fall2018/group1/linux-yocto-3.19\\
2). cp /scratch/files/config-3.19.2-yocto-standard .config\\
2). make -j4 all\\\\
{\bfseries Compiling Latex}\\
1). cd /scratch/fall2018/group1/Writeup1\\
2). make TRG=writeup1\\
\subsection{qemu Explanation}
{\bfseries Qemu Flags}\\
1). {\textbf{-S:}} Do not start the CPU at startup.\\
2). {\textbf{-nographic:}} Disable graphical output so that QEMU is a command line only application.\\
3). {\textbf{-kernel:}} Use a bzImage as the kernel, in our case it is using the Intel x86 architecture.\\
4). {\textbf{-drive file=core-image-lsb-sdk-qemux86.ext4,if=virtio:}} this is used to open an image used file descriptors\\
5). {\textbf{-enable-kvm:}} Allows full virtualization support.\\
6). {\textbf{-net none:}} There is no on-board NIC.\\
7). {\textbf{-usb:}} Enables USB driver.\\
8). {\textbf{-localtime:}} ??\\
9). {\textbf{--no-reboot:}} exit instead of rebooting\\
10). {\textbf{--append "root=/dev/vda rw console=ttyS0 debug":}} ??\\\\
\subsection{Concurrency}
Anton

\subsection{Version Control}

\subsection{Work Log}

\end{document}
