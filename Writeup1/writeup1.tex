\documentclass[onecolumn, draftclsnofoot, 10pt, compsoc]{IEEEtran}
\usepackage{graphicx}
\usepackage{url}
\usepackage{setspace}

\usepackage{amssymb}
\usepackage{amsmath}
\usepackage{amsthm}
\usepackage{alltt}
\usepackage{color}
\usepackage{enumitem}
\usepackage{textcomp}
\usepackage{cite}

\usepackage{lmodern}
\usepackage[hidelinks]{hyperref}
\usepackage[normalem]{ulem}

\usepackage[utf8]{inputenc}
\usepackage[english]{babel}
\usepackage{lineno}
\usepackage{upquote}
\usepackage{fvextra}
\usepackage{fancyvrb}
\usepackage{minted}
\usepackage{longtable,hyperref}

\usepackage{geometry}
\geometry{textheight=9.5in, textwidth=7in}

\parindent = 0.0 in
\parskip = 0.1 in

%pull in the necessary preamble matter for pygments output
%\usepackage{fancyvrb}
\usepackage{color}
\usepackage[latin1]{inputenc}


\makeatletter
\def\PY@reset{\let\PY@it=\relax \let\PY@bf=\relax%
    \let\PY@ul=\relax \let\PY@tc=\relax%
    \let\PY@bc=\relax \let\PY@ff=\relax}
\def\PY@tok#1{\csname PY@tok@#1\endcsname}
\def\PY@toks#1+{\ifx\relax#1\empty\else%
    \PY@tok{#1}\expandafter\PY@toks\fi}
\def\PY@do#1{\PY@bc{\PY@tc{\PY@ul{%
    \PY@it{\PY@bf{\PY@ff{#1}}}}}}}
\def\PY#1#2{\PY@reset\PY@toks#1+\relax+\PY@do{#2}}

\expandafter\def\csname PY@tok@gd\endcsname{\def\PY@tc##1{\textcolor[rgb]{0.63,0.00,0.00}{##1}}}
\expandafter\def\csname PY@tok@gu\endcsname{\let\PY@bf=\textbf\def\PY@tc##1{\textcolor[rgb]{0.50,0.00,0.50}{##1}}}
\expandafter\def\csname PY@tok@gt\endcsname{\def\PY@tc##1{\textcolor[rgb]{0.00,0.25,0.82}{##1}}}
\expandafter\def\csname PY@tok@gs\endcsname{\let\PY@bf=\textbf}
\expandafter\def\csname PY@tok@gr\endcsname{\def\PY@tc##1{\textcolor[rgb]{1.00,0.00,0.00}{##1}}}
\expandafter\def\csname PY@tok@cm\endcsname{\let\PY@it=\textit\def\PY@tc##1{\textcolor[rgb]{0.25,0.50,0.50}{##1}}}
\expandafter\def\csname PY@tok@vg\endcsname{\def\PY@tc##1{\textcolor[rgb]{0.10,0.09,0.49}{##1}}}
\expandafter\def\csname PY@tok@m\endcsname{\def\PY@tc##1{\textcolor[rgb]{0.40,0.40,0.40}{##1}}}
\expandafter\def\csname PY@tok@mh\endcsname{\def\PY@tc##1{\textcolor[rgb]{0.40,0.40,0.40}{##1}}}
\expandafter\def\csname PY@tok@go\endcsname{\def\PY@tc##1{\textcolor[rgb]{0.50,0.50,0.50}{##1}}}
\expandafter\def\csname PY@tok@ge\endcsname{\let\PY@it=\textit}
\expandafter\def\csname PY@tok@vc\endcsname{\def\PY@tc##1{\textcolor[rgb]{0.10,0.09,0.49}{##1}}}
\expandafter\def\csname PY@tok@il\endcsname{\def\PY@tc##1{\textcolor[rgb]{0.40,0.40,0.40}{##1}}}
\expandafter\def\csname PY@tok@cs\endcsname{\let\PY@it=\textit\def\PY@tc##1{\textcolor[rgb]{0.25,0.50,0.50}{##1}}}
\expandafter\def\csname PY@tok@cp\endcsname{\def\PY@tc##1{\textcolor[rgb]{0.74,0.48,0.00}{##1}}}
\expandafter\def\csname PY@tok@gi\endcsname{\def\PY@tc##1{\textcolor[rgb]{0.00,0.63,0.00}{##1}}}
\expandafter\def\csname PY@tok@gh\endcsname{\let\PY@bf=\textbf\def\PY@tc##1{\textcolor[rgb]{0.00,0.00,0.50}{##1}}}
\expandafter\def\csname PY@tok@ni\endcsname{\let\PY@bf=\textbf\def\PY@tc##1{\textcolor[rgb]{0.60,0.60,0.60}{##1}}}
\expandafter\def\csname PY@tok@nl\endcsname{\def\PY@tc##1{\textcolor[rgb]{0.63,0.63,0.00}{##1}}}
\expandafter\def\csname PY@tok@nn\endcsname{\let\PY@bf=\textbf\def\PY@tc##1{\textcolor[rgb]{0.00,0.00,1.00}{##1}}}
\expandafter\def\csname PY@tok@no\endcsname{\def\PY@tc##1{\textcolor[rgb]{0.53,0.00,0.00}{##1}}}
\expandafter\def\csname PY@tok@na\endcsname{\def\PY@tc##1{\textcolor[rgb]{0.49,0.56,0.16}{##1}}}
\expandafter\def\csname PY@tok@nb\endcsname{\def\PY@tc##1{\textcolor[rgb]{0.00,0.50,0.00}{##1}}}
\expandafter\def\csname PY@tok@nc\endcsname{\let\PY@bf=\textbf\def\PY@tc##1{\textcolor[rgb]{0.00,0.00,1.00}{##1}}}
\expandafter\def\csname PY@tok@nd\endcsname{\def\PY@tc##1{\textcolor[rgb]{0.67,0.13,1.00}{##1}}}
\expandafter\def\csname PY@tok@ne\endcsname{\let\PY@bf=\textbf\def\PY@tc##1{\textcolor[rgb]{0.82,0.25,0.23}{##1}}}
\expandafter\def\csname PY@tok@nf\endcsname{\def\PY@tc##1{\textcolor[rgb]{0.00,0.00,1.00}{##1}}}
\expandafter\def\csname PY@tok@si\endcsname{\let\PY@bf=\textbf\def\PY@tc##1{\textcolor[rgb]{0.73,0.40,0.53}{##1}}}
\expandafter\def\csname PY@tok@s2\endcsname{\def\PY@tc##1{\textcolor[rgb]{0.73,0.13,0.13}{##1}}}
\expandafter\def\csname PY@tok@vi\endcsname{\def\PY@tc##1{\textcolor[rgb]{0.10,0.09,0.49}{##1}}}
\expandafter\def\csname PY@tok@nt\endcsname{\let\PY@bf=\textbf\def\PY@tc##1{\textcolor[rgb]{0.00,0.50,0.00}{##1}}}
\expandafter\def\csname PY@tok@nv\endcsname{\def\PY@tc##1{\textcolor[rgb]{0.10,0.09,0.49}{##1}}}
\expandafter\def\csname PY@tok@s1\endcsname{\def\PY@tc##1{\textcolor[rgb]{0.73,0.13,0.13}{##1}}}
\expandafter\def\csname PY@tok@sh\endcsname{\def\PY@tc##1{\textcolor[rgb]{0.73,0.13,0.13}{##1}}}
\expandafter\def\csname PY@tok@sc\endcsname{\def\PY@tc##1{\textcolor[rgb]{0.73,0.13,0.13}{##1}}}
\expandafter\def\csname PY@tok@sx\endcsname{\def\PY@tc##1{\textcolor[rgb]{0.00,0.50,0.00}{##1}}}
\expandafter\def\csname PY@tok@bp\endcsname{\def\PY@tc##1{\textcolor[rgb]{0.00,0.50,0.00}{##1}}}
\expandafter\def\csname PY@tok@c1\endcsname{\let\PY@it=\textit\def\PY@tc##1{\textcolor[rgb]{0.25,0.50,0.50}{##1}}}
\expandafter\def\csname PY@tok@kc\endcsname{\let\PY@bf=\textbf\def\PY@tc##1{\textcolor[rgb]{0.00,0.50,0.00}{##1}}}
\expandafter\def\csname PY@tok@c\endcsname{\let\PY@it=\textit\def\PY@tc##1{\textcolor[rgb]{0.25,0.50,0.50}{##1}}}
\expandafter\def\csname PY@tok@mf\endcsname{\def\PY@tc##1{\textcolor[rgb]{0.40,0.40,0.40}{##1}}}
\expandafter\def\csname PY@tok@err\endcsname{\def\PY@bc##1{\setlength{\fboxsep}{0pt}\fcolorbox[rgb]{1.00,0.00,0.00}{1,1,1}{\strut ##1}}}
\expandafter\def\csname PY@tok@kd\endcsname{\let\PY@bf=\textbf\def\PY@tc##1{\textcolor[rgb]{0.00,0.50,0.00}{##1}}}
\expandafter\def\csname PY@tok@ss\endcsname{\def\PY@tc##1{\textcolor[rgb]{0.10,0.09,0.49}{##1}}}
\expandafter\def\csname PY@tok@sr\endcsname{\def\PY@tc##1{\textcolor[rgb]{0.73,0.40,0.53}{##1}}}
\expandafter\def\csname PY@tok@mo\endcsname{\def\PY@tc##1{\textcolor[rgb]{0.40,0.40,0.40}{##1}}}
\expandafter\def\csname PY@tok@kn\endcsname{\let\PY@bf=\textbf\def\PY@tc##1{\textcolor[rgb]{0.00,0.50,0.00}{##1}}}
\expandafter\def\csname PY@tok@mi\endcsname{\def\PY@tc##1{\textcolor[rgb]{0.40,0.40,0.40}{##1}}}
\expandafter\def\csname PY@tok@gp\endcsname{\let\PY@bf=\textbf\def\PY@tc##1{\textcolor[rgb]{0.00,0.00,0.50}{##1}}}
\expandafter\def\csname PY@tok@o\endcsname{\def\PY@tc##1{\textcolor[rgb]{0.40,0.40,0.40}{##1}}}
\expandafter\def\csname PY@tok@kr\endcsname{\let\PY@bf=\textbf\def\PY@tc##1{\textcolor[rgb]{0.00,0.50,0.00}{##1}}}
\expandafter\def\csname PY@tok@s\endcsname{\def\PY@tc##1{\textcolor[rgb]{0.73,0.13,0.13}{##1}}}
\expandafter\def\csname PY@tok@kp\endcsname{\def\PY@tc##1{\textcolor[rgb]{0.00,0.50,0.00}{##1}}}
\expandafter\def\csname PY@tok@w\endcsname{\def\PY@tc##1{\textcolor[rgb]{0.73,0.73,0.73}{##1}}}
\expandafter\def\csname PY@tok@kt\endcsname{\def\PY@tc##1{\textcolor[rgb]{0.69,0.00,0.25}{##1}}}
\expandafter\def\csname PY@tok@ow\endcsname{\let\PY@bf=\textbf\def\PY@tc##1{\textcolor[rgb]{0.67,0.13,1.00}{##1}}}
\expandafter\def\csname PY@tok@sb\endcsname{\def\PY@tc##1{\textcolor[rgb]{0.73,0.13,0.13}{##1}}}
\expandafter\def\csname PY@tok@k\endcsname{\let\PY@bf=\textbf\def\PY@tc##1{\textcolor[rgb]{0.00,0.50,0.00}{##1}}}
\expandafter\def\csname PY@tok@se\endcsname{\let\PY@bf=\textbf\def\PY@tc##1{\textcolor[rgb]{0.73,0.40,0.13}{##1}}}
\expandafter\def\csname PY@tok@sd\endcsname{\let\PY@it=\textit\def\PY@tc##1{\textcolor[rgb]{0.73,0.13,0.13}{##1}}}

\def\PYZbs{\char`\\}
\def\PYZus{\char`\_}
\def\PYZob{\char`\{}
\def\PYZcb{\char`\}}
\def\PYZca{\char`\^}
\def\PYZam{\char`\&}
\def\PYZlt{\char`\<}
\def\PYZgt{\char`\>}
\def\PYZsh{\char`\#}
\def\PYZpc{\char`\%}
\def\PYZdl{\char`\$}
\def\PYZti{\char`\~}
% for compatibility with earlier versions
\def\PYZat{@}
\def\PYZlb{[}
\def\PYZrb{]}

% these were missing from downloaded version
\def\PYZhy{-}
\def\PYZdq{"} 
\makeatother


\newcommand{\longtableendfoot}{Please continue at the next page}

\begin{document}
\begin{titlepage}
\pagenumbering{gobble}
\begin{singlespace}
\centering
\scshape{
    \huge{Writeup 1}\par
    \vspace{.5in}
    \large{CS 444}\par
    \large{October 29, 2018}\par
    \vspace{.5in}
    \large{Anton Synytsia, Eytan Brodsky, David Jansen}\par
    \vspace{.5in}
    \vfill
}
%\begin{abstract}
%\end{abstract}
\end{singlespace}
\end{titlepage}
\newpage
\pagenumbering{arabic}
\tableofcontents
\clearpage

\section{Command Log}
{\bfseries Logging into OS2}
\begin{enumerate}
\item ssh os2.engr.oregonstate.edu
\end{enumerate}
{\bfseries Creating Group Folder}
\begin{enumerate}
\item cd /scratch/fall/2018
\item mkdir group1
\end{enumerate}
{\bfseries Cloning Repository}
\begin{enumerate}
\item cd /scratch/fall2018/group1/
\item git clone git://git.yoctoproject.org/linux-yocto-3.19
\item git checkout tags/v3.19.2
\end{enumerate}
{\bfseries Copying Files}
\begin{enumerate}
\item cp -R /scratch/files. /scratch/fall2018/group1
\end{enumerate}
{\bfseries Setting Up Environment}
\begin{enumerate}
\item cd /scratch/fall2018/group1/
\item source environment-setup-i586-poky-linux
\item qemu-system-i386 -gdb tcp::5501 -S -nographic -kernel bzImage-qemux86.bin -drive file=core-image-lsb-sdk-qemux86.ext4,if=virtio -enable-kvm -net none -usb -localtime
--no-reboot --append "root=/dev/vda rw console=ttyS0 debug"
\end{enumerate}
This will hang, which it should; to unhang the process, follow the steps below.

{\bfseries Debugging}
\begin{enumerate}
\item Start another terminal in OS2 and run the following commands:
\item gdb
\item target remote tcp::5501
\item continue
\end{enumerate}
{\bfseries Part 2: Testing Toolchain}
\begin{enumerate}
\item cd /scratch/fall2018/group1/linux-yocto-3.19
\item cp /scratch/files/config-3.19.2-yocto-standard .config
\item make -j4 all
\end{enumerate}
{\bfseries Compiling Writeup}
\begin{enumerate}
\item cd /scratch/fall2018/group1/Writeup1
\item make
\end{enumerate}
\section{qemu Explanation}
{\bfseries Qemu Flags}
\begin{enumerate}
\item {\textbf{-S:}} Do not start the CPU at startup.
\item {\textbf{-nographic:}} Disable graphical output so that QEMU is a command line only application.
\item {\textbf{-kernel:}} Use a bzImage as the kernel, in our case it is using the Intel x86 architecture.
\item {\textbf{-drive file=core-image-lsb-sdk-qemux86.ext4,if=virtio:}} this is used to open an image used file descriptors
\item {\textbf{-enable-kvm:}} Allows full virtualization support.
\item {\textbf{-net none:}} There is no on-board NIC.
\item {\textbf{-usb:}} Enables USB driver.
\item {\textbf{-localtime:}} Legacy option that's currently undocumented. Replaced by -rtc "localtime", which lets the TRC start at the current UTC time.
\item {\textbf{--no-reboot:}} exit instead of rebooting
\item {\textbf{--append "root=/dev/vda rw console=ttyS0 debug":}} Enables debug text to display on the user's terminal
\end{enumerate}

\section{Concurrency}
\begin{enumerate}
\item The objective of concurrency is to understand how to synchronize operations between two tasks.
\item We use pthreads for multi-threading and mutexes for critical sections. One buffer, two threads (plus the main thread), and one mutex are involved in producing and consuming jobs. Details regarding the role of each are described below:
\begin{description}
\item[Buffer] The buffer is a circular FIFO queue, containing jobs awaiting to be consumed. The buffer's starting and ending points are marked by \texttt{jhead} and \texttt{jtail} variables. Produced jobs are appended to the circular queue, incrementing and wrapping \texttt{jtail}. Consumed jobs are removed out of the circular queue, incrementing and wrapping \texttt{jhead}.

The buffer has a limit defined by \texttt{MAX\_JOBS} preprocessor. Once \texttt{jtail} reaches the limit, that is one less than \texttt{jhead}, the producer stops creating jobs until space becomes available again.
\item[Producer] The producer thread is a continuous while loop, which does the following:
\begin{enumerate}
\item Acquire mutex via \texttt{pthread\_mutex\_lock}.
\item Check if the jobs queue is not full. If the queue is full, simply unlock the mutex and continue the loop.
\item Create a job with a random wait time (2 to 9 seconds) and a random number, between 0 and 1000. The randomization is achieved by \texttt{\_rdrand32\_step} intrinsic if supported or \texttt{mt19937ar.c} if not.
\item Append job to the queue, incrementing and wrapping \texttt{jtail}.
\item Release the mutex via \texttt{pthread\_mutex\_unlock}.
\item Sleep 3 to 7 seconds.
\end{enumerate}
\item[Consumer] The consumer thread is also a continuous while loop, which does the following:
\begin{enumerate}
\item Acquire mutex.
\item Check if queue is not empty. If empty, release the mutex and continue the loop.
\item Get job at \texttt{jhead}.
\item Print out the number value of job.
\item Copy job wait time to a separate variable, \texttt{dt}.
\item Increment and wrap \texttt{jhead}.
\item Release the mutex.
\item Sleep for a wait time indicated at \texttt{dt}.
\end{enumerate}
\end{description}
\item In order to validate concurrency, we printed out every time a job was produced and every time a job was about to be consumed. A 3-7 delay  added after producing a job, to ensure that a job can be consumed while jobs are being produced. Once the buffer fills up, the producer waits for consumer to consume a job.
\item Although I, Anton, had previous experience with mutexes and multi-threading, I did learn that concurrency can as well be achieved with semaphores.
\end{enumerate}

\section{Version Control}
% This file was generated by the script latex-git-log
\begin{tabular}{lp{12cm}}
  \label{tabular:legend:git-log}
  \textbf{acronym} & \textbf{meaning} \\
  V & \texttt{version} \\
  tag & \texttt{git tag} \\
  MF & Number of \texttt{modified files}. \\
  AL & Number of \texttt{added lines}. \\
  DL & Number of \texttt{deleted lines}. \\
\end{tabular}

\bigskip

%\iflanguage{ngerman}{\shorthandoff{"}}{}
\begin{longtable}{|rllp{8.5cm}rrr|}
\hline \multicolumn{1}{|c}{\textbf{V}} & \multicolumn{1}{c}{\textbf{tag}}
& \multicolumn{1}{c}{\textbf{date}}
& \multicolumn{1}{c}{\textbf{commit message}} & \multicolumn{1}{c}{\textbf{MF}}
& \multicolumn{1}{c}{\textbf{AL}} & \multicolumn{1}{c|}{\textbf{DL}} \\ \hline
\endhead

\hline \multicolumn{7}{|r|}{\longtableendfoot} \\ \hline
\endfoot

\hline% \hline
\endlastfoot

\hline 1 &  & 2018-10-08 & Initial commit & 1 & 1 & 0 \\
\hline 2 &  & 2018-10-08 & Create .gitignore & 1 & 6 & 0 \\
\hline 3 &  & 2018-10-08 & Update state & 1 & 0 & 1 \\
\hline 4 &  & 2018-10-08 & Setup readme & 1 & 5 & 0 \\
\hline 5 &  & 2018-10-08 & Update README.md & 1 & 2 & 0 \\
\hline 6 &  & 2018-10-08 & Uploaded concurrency & 2 & 225 & 0 \\
\hline 7 &  & 2018-10-08 & changed permissions & 4 & 0 & 0 \\
\hline 8 &  & 2018-10-09 & Setup writeup TeX & 2 & 126 & 0 \\
\hline 9 &  & 2018-10-09 & Added pygments & 1 & 98 & 0 \\
\hline 10 &  & 2018-10-09 & included pygements to preamble & 1 & 8 & 6 \\
\hline 11 &  & 2018-10-09 & First sucessful compile of writeup1 & 6 & 2008 & 2 \\
\hline 12 &  & 2018-10-09 & Updated gitignore & 5 & 3 & 2005 \\
\hline 13 &  & 2018-10-09 & Setup Writeup & 1 & 10 & 20 \\
\hline 14 &  & 2018-10-09 & Added git attributes to enforce EOL & 1 & 7 & 0 \\
\hline 15 &  & 2018-10-09 & EOL deal & 2 & 1 & 2 \\
\hline 16 &  & 2018-10-09 & Trying to fix TEX compiling & 2 & 79 & 1 \\
\hline 17 &  & 2018-10-09 & Added sections to writeup & 3 & 8 & 81 \\
\hline 18 &  & 2018-10-09 & Added the writeup for Command Logs and for the Qemu flags & 11 & 97 & 10 \\
\hline 19 &  & 2018-10-10 & Work on concurrency writeup & 8 & 30 & 56 \\
\hline 20 &  & 2018-10-10 & More work on Concurrency writeup & 2 & 12 & 7 \\
\hline 21 &  & 2018-10-10 & Made Concurrency use circular queue & 1 & 41 & 34 \\
\hline 22 &  & 2018-10-10 & Fixed concurrency print type & 1 & 9 & 6 \\
\hline 23 &  & 2018-10-10 & Renamed Assignment1 folder to Concurency and finished my writeup & 7 & 251 & 244 \\
\hline 24 &  & 2018-10-11 & Disabled BIB command in makefile & 2 & 3 & 3 \\
\hline 25 &  & 2018-10-11 & Added compile instructions & 1 & 26 & 3 \\
\hline 26 &  & 2018-10-11 & README adjustments & 1 & 3 & 3 \\
\hline 27 &  & 2018-10-13 & corrected name & 1 & 1 & 1 \\
\hline 28 &  & 2018-10-13 & explain -localtime and -append root=/dev/vda rw console=ttyS0 debug & 1 & 2 & 2 \\
\hline 29 &  & 2018-10-13 & add work log. & 1 & 6 & 0 \\
\hline 30 &  & 2018-10-13 & Added the version control table & 1 & 31 & 1 \\
\hline 31 &  & 2018-10-13 & Changed up how the lists were made & 1 & 48 & 31 \\
\hline 32 &  & 2018-10-15 & Refactored rdrand to use X86 intrinsic and created a general rand function & 1 & 88 & 69 \\
\hline 33 &  & 2018-10-15 & Updated concurrency writeup to reflect the refactored code. & 1 & 7 & 10 \\
\end{longtable}


\section{Work Log}
\begin{description}
    \item[Getting Acquainted]
    The "Getting Acquainted" section was broken up between the group members to more evenly balance the work. In week two we started and completed the core of the assignment, which was to set up a working version of the yocto kernel where we tested both the emulator and the toolchain. This process was faster than expected since everything surprisingly just worked. At the beginning of week three we started working on the write up, where each person was assigned a specific section to complete sometime before the due date.
    \item[Concurrency]
    We completed the "Concurrency" section on week two while at the library. This took a couple of hours to write a working prototype, debug it, and add specific features such as buffer limits, addition speed, etc... mainly for verifying functionality. We followed a loose group coding technique where one person would write the code and the other would make suggestions, corrections, and look up different documentations as needed. This made the coding process go very smoothly and we were able to finish relatively quickly.
\end{description}

\end{document}
